%% Tested html conversion on https://ngpdf.com, it looks ok, and checked in
%% its editor.  Result seems as expected, because the
%% \etocthetocitembegintag/\etocthetocitemendtag create one top Reference
%% struct all around.

%% This made me discover that I actually don't need \etoctagReference around
%% an \etocthelinkedname for example.  The \etoctagReference records its
%% argument in the big surrounding Reference struct, but already the linking
%% is enough to interrupt temporarily the "artifact=layout" MC I am a bit
%% worried though in ngpdf.com about the boxes shown in the two panels view.

% This is a variation on the test_tagging_toc_as_molecule.tex
\DocumentMetadata{
 uncompress,
 pdfversion=1.7,
 lang=en-US,
 testphase=phase-III
}
\ExplSyntaxOn
 \msg_redirect_module:nnn { tag } { warning } { error }
\ExplSyntaxOff
\documentclass{article}
%\usepackage{tagpdf-debug}
\usepackage[dvipsnames]{xcolor}
\usepackage{tikz}
\usetikzlibrary{trees}
\usepackage{hyperref}
\usepackage{etoc}
\usepackage{centeredline}
\usepackage{ragged2e}
\begin{document}

%% Adapted from file `etocsnippet-17.tex' of etoc.pdf 1.2d 2023/10/29
%% but for section>subsection rather than subsection > subsubsection
%% Also serves as test for the new possiblity to use \etocthelinkedname etc..
%% inside an \edef (meaning, the \hyperlink there-in is now \protected,
%% but things inside \etocthename possibly still do not like \edef,
%% of course for \etocthenumber and \etocthepage, no problem is expected).

\newtoks\treetok % put this (uncommented) preferably in the preamble
\newtoks\sectiontok
\newtoks\subsectiontok

%% auxiliaries
% expands 2nd argument (macro) and appends it to 1st argument (toks)
% used below only once in \etocsettocstyle
\newcommand*\appendtotok[2]{% #1=toks variable, #2=macro, expands once #2
  #1\expandafter\expandafter\expandafter
    {\expandafter\the\expandafter #1#2}}

% appends 2nd argument contents (toks) as child of first argument (toks)
% used in the \etocsetlinestyle configurations
\newcommand*{\appendchildtree}[2]{% token list t1 becomes: t1 child {t2}
   \edef\tmp{\the#1 child {\the#2}}%
   #1\expandafter{\tmp}%
}

\newcommand*{\treenode}{}% only to make sure our \edef's do not overwrite
                         % an existing command


%% core handling of the toc line data

%% this macro is responsible also for injecting the tagging, it uses for this
%% conveniences from etoc but user is free to use directly tagpdf interface
%% (make sure to have checked etoc source code first to understand what is
%% happening here).

% This one sets the node, it is the core macro inserting the tagging code as
% well; some expansion is mandatory because \etocthetocitembegintag and
% \etocthetocitemendtag contain information such as whether being located in a
% section or subsection, information which is known only at the time of the
% processing of the toc line by the user defined etoc style.
\newcommand*{\preparetreenodeastaggedtocitem}{%
%% watch out the \edef! It is crucial to not make a \def here!
%%
%% MEMO: **really** I should use \protected@edef rather than the \noexpand's,
%% but I need to make @ a letter then...
%%
%% MEMO: I tried to use the TikZ keys [align=center, text width=2cm] but then
%% tagpdf complains Parent-Child 'Document/' --> 'TOCI/pdf'
%%
%% MEMO: I tried using a \parbox but it creates seemingly a MC.
%% so using \vbox...
%%
%% MEMO: I was originally using \etoctagReference but as the argument
%% always is hyperlinked it has seemingly no impact on the final tree
%% I see in the ngpdf.com editor
  \edef\treenode{node {\etocthetocitembegintag
%%                             do not forget vvvvvvvvv this here and elsewhere
                       \vbox{\hsize 2.5cm\relax\noexpand\Centering
                         \noexpand\textbf{\etocthelinkedname}
%% Anything not inside \etoctagReference or \etoctagLbl is automatically
%% tagged as artifact
%% Using here Lbl directly under TOCI, not nested in \etoctagReference
%% but it maybe it would be better to nest
                         (\noexpand\textit{\etoctagLbl{\etocthelinkednumber}})
%% \etocthelinkedpage carries hyperlink even if hyperref is left to its
%% default to not link the page number
                         on page \etocthelinkedpage
                       }%
                       \etocthetocitemendtag}}%
}

% The main objective of these somewhat subtle constructs is to build up
% the expected by TikZ input tree in its syntax
\etocsetlinestyle{section}
  {\etocskipfirstprefix}
  {\appendchildtree\treetok\sectiontok}
  {\preparetreenodeastaggedtocitem
   \sectiontok\expandafter{\treenode}}
  {\appendchildtree\treetok\sectiontok}

\etocsetlinestyle{subsection}
  {\etocskipfirstprefix}
  {\appendchildtree\sectiontok\subsectiontok}
  {\preparetreenodeastaggedtocitem
   \subsectiontok\expandafter{\treenode}}
  {\appendchildtree\sectiontok\subsectiontok}

% Finally here we have the core code to create and finish the \treetok
% we need a global in the "after" code because there will be a grouping
% created always by \tableofcontents
%%
%% MEMO: I tried using here  \node [align=center, text width=2cm]
%% but \tagstop was too late from inside the node
%% and moving it outside the node made TikZ "give up on the path"
%%
%% MEMO: I also tried things with opening a TOCI struct but never
%% could make it work seems I am lacking stuff from latex-lab such as
%% \seq_gpush:Nx \g_@@_toc_stack_seq {{TOCI}\use:c{toclevel@#1}} ???
\etocsettocstyle
  {\treetok{\node {\tagstop %% TODO: THINK OF A BETTER WAY TO HANDLE THIS
                   \parbox{2cm}{\raggedright THE eTOC AS A TIKZ MOLECULE}%
                   \tagstart
                   }%
           }%
  }
  {\global\appendtotok\treetok{ ;}}%
  

\centeredline{% from package centeredline 
% btw, the \centeredline creates a scope limiting group
   \etocsetnexttocdepth{subsection}%
% this tells etoc not add a \par
   \etocinline
% we turn the etoc tagging related actions off because the toc lines
% at this stage are only executed to accumulate tokens inside the \toks
   \etoctaggingoff  %% <<<< very important!
% this only parses the .toc file data to prepare suitably tokens
% to feed to TikZ in a second step
   \tableofcontents
%% it is not needed to do \etoctaggingon afterwards, see comment below.
%
% The \treetok will now contain all the tagging code ready to be
% executed, see the core \preparetreenodeastocitem macro above.
   \hypersetup{hidelinks}%
   \begin{tikzpicture}
              [grow cyclic,
               level 1/.style={level distance=5cm,sibling angle=72},
               level 2/.style={level distance=3cm,sibling angle=60},
               every node/.style={ball color=red!70,circle,text=white},
               edge from parent path={[dashed,very thick,color=cyan]
                           (\tikzparentnode) --(\tikzchildnode)}]
% This imitates the kernel tagging additions to the \@starttoc
% command; no argument, uses `toc' for the title key of the TOC structure
% MEMO: this is immune to \etoctaggingoff, as are \etoctagReference,
% \etoctagLbl, \etocthetocitembegintag, and \etocthetocitemendtag
% because \etoctaggingon/\etoctaggingoff is only related to the
% *automatic* tagging done by etoc.  Here we are using user-added
% manual tagging, using etoc helpers.
\etoctagstartTOCcontents
   \the\treetok
\etoctagfinishTOCcontents
   \end{tikzpicture}%
}

%\ShowTagging{debug/structures=2}

\section{One}

Texte

\subsection{One one}

Texte

\clearpage

\subsection{One two}

Texte

\clearpage

\subsection{One three}

Texte

\subsection{One four}

Texte

\clearpage

\section{Two}

\subsection{Two one}

Texte
\clearpage

\subsection{Two two}

Texte

\clearpage

\section{Three}

\subsection{Three one}

Texte

\subsection{Three two}

Texte

\clearpage
\subsection{Three three}

Texte

\end{document}
