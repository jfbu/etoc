\documentclass{article}
\usepackage[T1]{fontenc}
\usepackage[hscale=0.66,vscale=0.75]{geometry}
\usepackage{txfonts}
% Use the Bitter typeface for the romanfamily
\usepackage[scale=0.87]{bitter}

\AtBeginDocument{%
    \renewcommand{\familydefault}{\sfdefault}%
    \rmfamily
}

\usepackage{xspace}
\usepackage[dvipsnames,svgnames]{xcolor}
%\usepackage{graphicx}
\usepackage[inline]{enumitem}
\usepackage{setspace}
\usepackage{hyperref}
\hypersetup{%
  %linktoc=all,%
  breaklinks=true,%
  colorlinks,%
  linkcolor=RoyalBlue,% Orchid
  urlcolor=OliveGreen,%
  pdfauthor={Jean-François B.},% on peut y aller maintenant avec ç en 2022...
  pdftitle={WIP tagging with the etoc package},%
  pdfsubject={Tables of contents with LaTeX},%
  pdfkeywords={LaTeX, table of contents},%
  pdfstartview=FitH,%
  %%pdfpagemode=UseOutlines,
}
\usepackage{colorframed}% fix color issues with framed.sty (JFB 2022)
\colorlet{shadecolor}{gray!10}
\usepackage{centeredline}% custom macro now in public package

\DeclareRobustCommand\csa [1]
                {{\ttfamily\hyphenchar\font45 \char`\\ #1}}

% the real \csb is much more sophisticated... see etoc.dtx
\def\csb#1{\textcolor{RoyalBlue}{\csa{#1}}}

\newcommand\etoc{%
        \texorpdfstring{{\color{joli}\ttfamily\bfseries etoc}}{etoc}\xspace}
\newcommand\toc{\csb{tableofcontents}\xspace}
\newcommand\localtoc{\csb{localtableofcontents}\xspace}
\definecolor{joli}{RGB}{225,95,0}

\DeclareRobustCommand\ctanpkg[1]
      {\texorpdfstring{\href{https://ctan.org/pkg/#1}{#1}}{#1}}

\usepackage{doc}[=v2]
\providecommand\marg[1]{%
  {\ttfamily\char`\{}\meta{#1}{\ttfamily\char`\}}}
\providecommand\oarg[1]{%
  {\ttfamily[}\meta{#1}{\ttfamily]}}
\providecommand\parg[1]{%
  {\ttfamily(}\meta{#1}{\ttfamily)}}
\nonfrenchspacing


\begin{document}

(in the real document, the \textcolor{RoyalBlue}{blue} colored words are
doubly hyperlinked to their user manual definition and to their source code definition)

\onehalfspacing

\section{Tagging}
\label{etoctaggingon}
\label{etoctaggingoff}
\label{etoctagginginlineoff}

For some years upstream \LaTeX{} maintainers have been engaged into a
``\LaTeX{} Tagged PDF'' project to enable automatic tagging of PDF
documents produced via \LaTeX, see the relevant entries in
\centeredline{\url{https://latex-project.org/news/latex2e-news/ltnews.pdf}}
and a general overview in this 2020 paper
\centeredline{\url{https://www.latex-project.org/publications/2020-FMi-TUB-tb129mitt-tagpdf.pdf}}
See also further relevant articles in \href{https://tug.org/tugboat}{TUGboat}.

This means that there are ongoing kernel (and \ctanpkg{hyperref}) changes
which have to be taken into account by \etoc to not be broken, or cause a
broken PDF structure, once the user activates the feature, currently via usage of
suitable \csa{DocumentMetadata} keys, cf.\@ above documentation.

\begingroup\colorlet{shadecolor}{red!10}
\begin{shaded}
The author has neither the time nor the resources nor the appropriate tools to
do any kind of testing beyond the most basic by himself.

Besides he is mostly ignorant attow about tagging related matters.
\end{shaded}
\endgroup

Current status (may still refer to an earlier \texttt{etoc.sty}):
\begin{itemize}
\item A document using only \toc and \localtoc, with \etoc left in
  compatibility mode for line styles should be tagged as well as
  for documents not using \etoc facilities.  Indeed
  \etoc then does not at all interfere with the rendering of the
  \texttt{.toc} file data, apart from implementing the ``local toc''
  mechanism.  One should keep in mind though that:
  \begin{itemize}
  \item the tagging code from the kernel \csa{@starttoc} has been copied
    over manually by the author into \etoc source code, as it does not
    execute \csa{@starttoc}.  When upstream \LaTeX{} changes, \etoc has
    to be updated.
  \item if \csb{etocsettocstyle}, or higher level interface such as
    \csb{etocframedstyle}, has made \etoc leave the compatibility mode for the
    ``global display TOC style'', it may be that the user will have to add
    directly extra tagging code regarding the way for example the
    ``title'' of the TOC is rendered.
  \end{itemize}

\item A document using \csb{etocsetlinestyle} to define custom line
  styles using \csb{etocname}, \csb{etocnumber} and \csb{etocpage}
  should behave ``correctly''.  This has been tested with the own
  package ``fallback line styles'' (see \csb{etocfallbacklines}) which
  are but a special manner (actually quite poor, but that code was
  written at a time the author barely new \LaTeX), of using
  \csb{etocsetlinestyle}.  They needed no change.

  One should keep in mind though that:
  \begin{itemize}
  \item ``tested'' means here that the PDF was built (via PDF\LaTeX) and
    no warnings were reported by \ctanpkg{tagpdf}.  The author can not
    assess in any way whether the PDF are actually valid from the point
    of view of tagging.
  \item \emph{Anything apart from \emph{\csb{etocname}},
      \emph{\csb{etocnumber}} and \emph{\csb{etocpage}} from custom user
      line styles will by default be marked as being artifacts!}  If
    user needs to mark something otherwise, the current marked content
    chunk must be interrupted, the correct tagging added, and then the
    artifact restarted. Check the source code for how this is done
    (perhaps, probably, wrongly!) by
    the package for \csb{etocname} et al.
  \item the tagging code from the kernel \csa{contentsline} is recycled
    into \etoc source code, which does not use that macro if not in
    compatibility mode.  Each time upstream \LaTeX{} changes the way
    these code chunks are added to \csa{contentsline}, \etoc will need
    updating.
  \item the way \csb{etocname}, \csb{etocnumber} and \csb{etocpage} are
    tagged is imitated from kernel code, which currently uses the hook
    mechanism.  The hooks themselves are not reused by \etoc.  If and
    when the authors understand better the macros of \ctanpkg{tagpdf},
    perhaps changes will happen, but at least here, the \ctanpkg{tagpdf}
    API is used directly.
  \end{itemize}
\end{itemize}

\begingroup\colorlet{shadecolor}{red!10}
\begin{shaded}
  All user interface is to be considered currently unstable and may change at
any time.
\end{shaded}
\endgroup

\begin{description}
\item[\csb{etoctaggingon}] activates the tagging-related code from \etoc
  (this is done at begin document if document has activated tagging).  This is
  a no-op if document has not activated tagging.

  \csb{etocname}, \csb{etocnumber} and \csb{etocpage} will now accomplish
  tagging tasks, as will \csb{etoclink} and the non-robust \csb{etocthelink}.
  But \csb{etocthename}, \csb{etocthenumber}, and \csb{etocthepage} are kept
  as is, with no tagging.  This is especially important for \csb{etocthenumber}
  for backwards compatibility as it may be needed in some line styles as a
  numerical quantity.

\item[\csb{etoctaggingoff}] turns off (almost) all \etoc tagging-related
  code.

  As advanced \etoc users will be aware, it is possible with it to use
  the \toc and \localtoc commands to do things only remotely related to
  tables of contents per se, for example one can use it to count how
  many sections are contained in a chapter so the only result of
  execution of these commands is to update some counter or other
  user-defined macro to hold the result.  Use \csb{etoctaggingoff}
  before the \toc or \localtoc command then.

  Attow, this does \emph{not} deactivate turning off the \emph{minipage}
  tagging sockets, which currently seems required to fix
  \href{https://github.com/jfbu/etoc/issues/4}{etoc\#5}.  But in the use
  case from previous paragraph, this is irrelevant.

\item[\csb{etoctagginginlineoff}] is in case user employs
  \csb{etocsetlinestyle} to re-employ kernel macros such as
  |\@dottedtocline| or |\l@section|, see \autoref{sec:anothercompat}: it
  tells \etoc to free \csb{etocname}, \csb{etocnumber} and
  \csb{etocpage} from contributing any tagging data by themselves, as
  this will be inserted already by the used \LaTeX{} kernel macros where
  \csb{etocname} et al. have been reinserted.  It also turns off \etoc
  added hyperlinking because the kernel |\@dottedtocline| or
  |\l@section| will originate it by themselves if tagging has been
  activated in the document.

  This command (as \csb{etoctaggingoff}) is supposed to be used right
  before \toc or \localtoc.  It may (untested) also be used from inside
  the \marg{start} part of \csb{etocsetlinestyle} for
  a given sectioning name, it this is the only level using the \LaTeX{}
  kernel macros as per \autoref{sec:anothercompat}.  Then the
  \marg{finish} part may use \csb{etoctaggingon}.

  There is no \csa{etoctagginginlineon}, use \csb{etoctaggingon}.

\end{description}


WARNING: \emph{the next paragraph may now be partially obsolete due to non
  documented yet \csa{etocthetaggedname} etc..., but an interface is at
  any rate currently lacking to allow user
  to re-insert TOC and TOCI tags.}

Advanced usages of \etoc, such as those in this documentation which use \toc
or \localtoc to extract data from the \texttt{.toc} file, using
\csb{etocthename}, \csb{etocthenumber}, \csb{etocthepage}, which are then
rendered in a second stage, once the whole TOC has been parsed, via TikZ for
example, will need to use \csb{etoctaggingoff} locally and users are currently
on their own to add tagging manually to the TikZ (or whatever) rendering.

\section{Another compatibility mode}
\label{sec:anothercompat}

mock target for copied-pasted above text.
\end{document}